% -*- coding:utf-8 -*-
% vi:encoding=utf-8:
% !TEX encoding = UTF-8 Unicode
%
% This file should be utf8 encoded so that these characters render as
% umlauts: ÄÖÜäöüß
%
%
% FBR_Bericht_2025.tex
%
% Fachbeiratsbericht 2025
% 
% PHENO SECTION

\documentclass{FBR_Bericht_2025}

\setcounter{secnumdepth}{5}
\setcounter{tocdepth}{3}

%\renewcommand{\thechapter}{\Roman{chapter}}
%\renewcommand{\thesection}{\arabic{section}}

\usepackage{physics}

\usepackage[backend=biber,sorting=none]{biblatex}

\addbibresource{Pheno.bib}
\input{inc/definitions.tex}

\begin{document}

\onecolumn
\pdfbookmark[1]{Table of Contents}{table_of_contents}
\tableofcontents
\cleardoublepage

\twocolumn

% ----------------------------------------------------------------------
\chapter{Research Activities -- Theory}
% ----------------------------------------------------------------------
\section[Phenomenology]{Novel computational techniques in particle physics and phenomenological applications}
% ----------------------------------------------------------------------
\begin{Namen}
Director: Prof. Dr. G. Zanderighi
\end{Namen}
Introduction of the group.

%%%%%%%%%%%%%%%%%%%%%%%%%%%%%%%%%%%%%%%%%%%%%%%%%%%%%%%%%%%%%%%%%%%%%%
\subsection{Novel event simulations for the hadron--hadron collisions}
%%%%%%%%%%%%%%%%%%%%%%%%%%%%%%%%%%%%%%%%%%%%%%%%%%%%%%%%%%%%%%%%%%%%%%
\begin{refsection}
Full fledged simulations of hadron-level events build the theoretical 
core of any experimental analyses performed at colliders.
By closing the gap between the measured data in the experimental
detectors and the theoretical predictions based on quantum-field theory
(QFT) computations, event generators are one of central tools developed
in particle theory. In addition, their predictions provide a realistic description
of any infrared-safe observable in comparison to data.

To keep up with the steadily decreasing experimental uncertainties, it is 
mandatory to develop event simulations at the highest possible accuracy.
This can be achieved by improving the perturbative all-order description 
of parton showers, on the one hand, and by including higher-order corrections
in the event simulations. This will allows us ultimately to observe even smallest
deviations from SM predictions and paves the way towards new-physics phenomena.

Event generators with next-to-leading order (NLO) accuracy are the standard since 
several years. However, there has been a substantial progress in achieving next-to-NLO
(NNLO) event simulations in the past years, which have been driven by our research
group. The \minnlo{} method to match NNLO predictions with parton showers (NNLO+PS)
was introduced by us in 2019 for $2\to 1$ colour-singlet 
processes \cite{Monni2019:whf,MonniXXX}, generalized $2\to 2$ reactions 
(and beyond) \cite{Lombardi:} and even extended to top-quark pair ($t\bar t$) production \cite{Mazzitelli, Mazzitelli}. With the latter work \minnlo{} became the first 
(and still only) NNLO+PS method to deal with colour charges in initial and final state.
%
\subsubsection[NNLO+PS predictions for $B$-hadron and $b$-jet production]{NNLO+PS predictions for \boldmath{$B$}-hadron and \boldmath{$b$}-jet production}

\begin{Namen}
R. Gauld, J. Mazzitelli, A. Ratti, M. Wiesemann, G. Zanderighi
\end{Namen}

Based on the previous advancements of the \minnlo{} method,
we have implemented a new NNLO+PS generator for bottom-quark quark pair ($b\bar b$) production.
To this end, the previous \minnlo{} implementation in POWHEG-BOX-V2 was ported 
to the POWHEG-BOX-RES framework, and then extended to account for 
general quark masses as well as different flavour-number schemes by implementing
variable number of light quark flavours.


\begin{table}[b!]
\begin{center}
\includegraphics[width=1\linewidth]{plots/bb_table.pdf}
\caption{$B$-hadron cross sections in $\mu$b.}
\label{tab:bb}
\end{center}
\end{table}

The $b\bar b$ \minnlo{}  generator has been developed in \citere{Ratti:} and employed for 
realistic predictions of $B$-hadron production in comparison to various LHC measurements.
By and large, a remarkable description of the measured cross sections is achieved, 
see \tab{tab:bb} for instance. We are currently finalizing a second study where we 
consider $b$-jet observables using our $b\bar b$ \minnlo{} simulation as
well as massless (and massive) NLO+PS dijet simulations. In this context, 
algorithms to appropriately define the bottom flavour in jets play a crucial role, see
\citere{Rhorryspaper} for instance, and we find that the by far dominant effect 
of the difference with to the standard $b$-tagging algorithms used by the experiments
stems from the treatment of $g\to b\bar b$ splittings.
A preliminary comparison of our \minnlo{} predictions against ATLAS data is given in 
\fig{fig:bb}.

\begin{figure}[t]
\begin{center}
\includegraphics[width=0.95\linewidth]{plots/bjet_mbb.pdf}
\caption{Invariant-mass distribution of leading $b$-jets for \minnlo{} (blue), \minlo{} (grey).}
\label{fig:bb}
\end{center}
\end{figure}
%
\subsubsection[$D$-meson processes at NNLO+PS and prompt neutrino background]{\boldmath{$D$}-meson processes at NNLO+PS \mbox{and prompt neutrino background}}
\begin{Namen}
R. Gauld, T. Giani, A. Mahr, A. Ratti, M. Wiesemann, G. Zanderighi
\end{Namen}

Another important application of the developed \minnlo{} methology is the production
of a charm-quark pair ($c\bar c$) 
in hadron--hadron collisions. While this process also receives
substantial attention by the current LHC experiments, additionally it plays a crucial role in 
the envisaged forward-physics experiments, in particular FASER. Even more crucially,
it is source of a major background to neutrino telescopes, such as ICECUBE, namely
the atmospheric prompt neutrino flux.
In all cases, the underlying mechanism is to produce charm quarks through the collision
of hadrons (either in a controlled collider environment or through cosmic rays interactions 
with the atmosphere), and the charm quarks then hadronize to $D$-mesons. Finally, in 
their further decay process highly energetic neutrinos are produced.

We have implemented a \minnlo{} generator for $c\bar c$ production to compute
$D$-meson cross sections at NNLO+PS. Moreover, we have started to study 
the propagation of particles through the the atmosphere through cascade equation
based on the code Matrix Cascade Equations (MCEq) \cite{} to compute the prompt
atmospheric neutrino flux. In \fig{fig:neutrinoflux} the prompt atmospheric neutrino
flux is shown as a function of the neutrino energy comparing our LO and NLO+PS
inputs to the default MCEq results. As can be seen, the NLO corrections are substantial.
Moreover, the perturbative scale uncertainties (not shown here) are similarly large, 
jeopardizing the precision of the predictions for the neutrino flux. 
This is a severe limitation, which calls for the inclusion of higher-order corrections.
By upgrading these predictions with NNLO accuracy for $c\bar c$ production of 
our \minnlo{} generator we hope to significantly improve the estimation of the 
prompt atmospheric neutrino background in ICECUBE in terms of both accuracy and
precision.

\begin{figure}[h]
\begin{center}
\includegraphics[width=0.95\linewidth]{plots/cc_neutrino_flux}
\caption{Invariant-mass distribution of leading $b$-jets for \minnlo{} (blue), \minlo{} (grey).}
\label{fig:neutrinoflux}
\end{center}
\end{figure}

%
\subsubsection[\minnlo{} method for heavy-quark plus colour-singlet production and application to $b\bar{b}Z$ process]{\minnlo{} method for heavy-quark plus colour-singlet production and the \boldmath{$b\bar{b}Z$} process}
\begin{Namen}
M. Wiesemann
\end{Namen}
%
\subsubsection{Associated Higgs production with bottom quarks: a flavour-scheme study at NNLO+PS}
\begin{Namen}
C. Biello, R. Gauld, A. Sankar, M. Wiesemann, G. Zanderighi
\end{Namen}
Higgs production associated with bottom quarks (\bbH{}) is a rare production mode at the LHC, serving as an irreducible background in Higgs-pair searches and gaining potential enhancement in BSM scenarios. Precise predictions of this process are important not only for experimental comparisons but also as a theoretical laboratory for studying heavy-quark effects. In processes involving bottom quarks, predictions can be obtained using different approaches to mass effects. In the scheme with five massless flavours (5FS), the bottom-quark component of the proton is perturbatively generated, enabling the resummation of large logarithmic contributions in the bottom mass. This approach treats the bottom quark as a massless parton, allowing for efficient high-order calculations. In~\citere{Biello:2024vdh}, we presented the first NNLO+PS simulation of this process within the massless scheme. For a more detailed description of bottom-quark kinematics, an alternative approach considers the production of massive bottom quarks in the hard scattering process, along with the Higgs boson, with four massless flavours (4FS). While computationally more demanding, this approach retains all mass effects order by order in perturbative QCD. Working in the \minnlo{} framework, which enables NNLO-accurate predictions for heavy-quark pair production in generic kinematics, we accessed the NNLO corrections in the massive scheme and matched them with a parton shower simulation~\cite{Biello:2024pgo}. Strong tensions between the massless and massive schemes in \bbH{} were observed in the past at lower accuracy, whereas a better agreement is found between the two NNLO+PS predictions, as shown in~\fig{phenofig:bbH}. The two schemes capture complementary aspects of bottom-quark dynamics. A generalised mass-variable flavour number scheme is desirable to achieve precise predictions across the full phase space. Using the method developed in~\citere{Gauld:2021zmq}, we are investigating the isolation of mass power corrections within the massive scheme, aiming to incorporate them into the massless prediction. This would allow us to retain the logarithmic resummation benefits of the massless scheme while systematically accounting for missing power corrections.

\begin{figure}[phtb]
\begin{center}
\includegraphics[width=0.95\linewidth]{plots/bbH__ptHspectrum.pdf}
\caption{Transverse spectrum of the Higgs boson produced via the bottom-quark Yukawa interaction at NNLO+PS in the massless (5FS, blue, dotted) and massive (4FS, red, solid) schemes~\cite{Biello:2024pgo}, with an NNLL+NNLO resummation result in the massless scheme (green, dashed).}
\label{phenofig:bbH}
\end{center}
\end{figure}
%
\subsubsection{Off-shell top-quark pair production at NNLO+PS}
\begin{Namen}
C. Biello, C. Signorile-Signorile, M. Wiesemann, G. Zanderighi
\end{Namen}
%
\subsubsection[\minnlo{} method for jet processes using $N$-jettiness]{\minnlo{} method for jet processes using \boldmath{$N$}-jettiness}
\begin{Namen}
M. Ebert, M. Wiesemann, G. Zanderighi, S. Zanoli
\end{Namen}
%
\subsubsection{Inclusion of electroweak corrections in \minnlo{}}
\begin{Namen}
G. Pelliccioli, M. Wiesemann, G. Zanderighi, S. Zanoli
\end{Namen}
%
\subsubsection{Polarized NLO+PS predictions and quantum info}
\begin{Namen}
G. Pelliccioli, G. Zanderighi
\end{Namen}
%
\printbibliography[heading=subbibliography]
\end{refsection}

\subsection{Fixed-order predictions for the hadron--hadron collisions}
%%%%%%%%%%%%%%%%%%%%%%%%%%%%%%%%%%%%%%%%%%%%%%%%%%%%%%%%%%%%%%%%%%%%%%
\begin{refsection}
Section for predictions at hadron-hadron colliders at fixed order.
%
\subsubsection{Top-mass renormalization in ttbar @ NNLO}
\begin{Namen}
J. Mazzitelli
\end{Namen}
%
\subsubsection{Wbb @ NNLO}
\begin{Namen}
J. Mazzitelli
\end{Namen}
%
\subsubsection{Z+c-jet NNLO QCD}
\begin{Namen}
R. Gauld
\end{Namen}
%
\subsubsection{Off-shell NLO predictions in tt(+X) processes}
\begin{Namen}
G. Pelliccioli
\end{Namen}
%
\printbibliography[heading=subbibliography]
\end{refsection}


%%%%%%%%%%%%%%%%%%%%%%%%%%%%%%%%%%%%%%%%%%%%%%%%%%%%%%%%%%%%%%%%%%%%%%
\subsection[Pushing the precision in Higgs studies]{Pushing the precision in Higgs studies}
%%%%%%%%%%%%%%%%%%%%%%%%%%%%%%%%%%%%%%%%%%%%%%%%%%%%%%%%%%%%%%%%%%%%%%
\begin{refsection}
Space for a nice introduction.
%
\subsubsection{VBF $H\rightarrow b\bar{b}$ production}
\begin{Namen}
A. Behring, G. Zanderighi
\end{Namen}
%
\subsubsection{Two-loop amplitudes for Higgs plus jet}
\begin{Namen}
U. Haisch, M. Niggetiedt
\end{Namen}
%
\subsubsection{Exact top-quark mass dependence in Higgs production}
\begin{Namen}
M. Niggetiedt
\end{Namen}
%
\subsubsection{Higgs predictions with bottom-quark mass effects}
\begin{Namen}
M. Niggetiedt
\end{Namen}
%
\subsubsection{Next-to-soft threshold in \bbH{}}
\begin{Namen}
A. Sankar
\end{Namen}
%
\subsubsection{Rapidity distribution of pseudoscalar Higgs}
\begin{Namen}
A. Sankar
\end{Namen}
%
\subsubsection{Di-Higgs production at NNLO+PS}
\begin{Namen}
F. Garosi, S. Kumar, M. Wiesemann, G. Zanderighi
\end{Namen}
%
\printbibliography[heading=subbibliography]
\end{refsection}

%%%%%%%%%%%%%%%%%%%%%%%%%%%%%%%%%%%%%%%%%%%%%%%%%%%%%%%%%%%%%%%%%%%%%%
\subsection{Tools and methods for higher-order predictions}
%%%%%%%%%%%%%%%%%%%%%%%%%%%%%%%%%%%%%%%%%%%%%%%%%%%%%%%%%%%%%%%%%%%%%%
\begin{refsection}
Space for a nice introduction.
%
\subsubsection{LASS: a subtraction scheme method at NNLO}
\begin{Namen}
G. Pelliccioli, A. Ratti, C. Signorile-Signorile
\end{Namen}
Include formulation and Strongly-ordered infrared counterterms from factorisation.
%
\subsubsection{New formulation of Nested Soft-Collinear Subtraction Scheme}
\begin{Namen}
C. Signorile-Signorile
\end{Namen}
%
\subsubsection{Four-loop renormalisation of pseudoscalar operators}
\begin{Namen}
M. Niggetiedt
\end{Namen}
%
\subsubsection{Soft function at N3LO}
\begin{Namen}
M. Delto, C. Wang
\end{Namen}
%
\subsubsection{Reclassifying Feynman Integrals as Special Functions}
\begin{Namen}
C. Wang
\end{Namen}
%
\printbibliography[heading=subbibliography]
\end{refsection}

%%%%%%%%%%%%%%%%%%%%%%%%%%%%%%%%%%%%%%%%%%%%%%%%%%%%%%%%%%%%%%%%%%%%%%
\subsection{Not only proton-proton collisions}
%%%%%%%%%%%%%%%%%%%%%%%%%%%%%%%%%%%%%%%%%%%%%%%%%%%%%%%%%%%%%%%%%%%%%%
\begin{refsection}
The application and development of new computational methods and simulation tools has also been pursued for different collider environments, such as those from lepton-hadron or lepton-lepton collisions.
%
Scattering processes with an initial-state lepton or leptons introduce different theoretical challenges as compared to that of hadron-hadron collisions, but also offer exciting opportunities to study the properties of the strong force in a relatively clean environment.
%
Such developments allow to re-visit the available data from colliders such as HERA and LEP with more precise theoretical simulations and inputs, increase the precision of simulation tools available for active experiments (e.g. IceCube and KM3NeT), and also provide a new level of theoretical precision for future experiments such as the EIC or a potential high-energy $e^+e^-$ collider. 
%
\subsubsection{NNLO+PS prediction for di-jet production at lepton colliders}
\begin{Namen}
F. Koenig, R. Schorer, M. Wiesemann, G. Zanderighi
\end{Namen}
%
\subsubsection{NLO+PS predictions for charged-lepton and neutrino induced DIS}
\begin{Namen}
R. Gauld, G. Zanderighi
\end{Namen}
The POWHEG method of matching fixed-order predictions with parton showers at NLO in QCD has been successfully applied to hadron-hadron collisions for a vast range of processes. As a consequence of the genuinely different kinematics encountered in lepton-hadron collisions, substantial extensions to the POWHEX BOX framework had to be undertaken to enable this method to be applied in this new collision environment~\cite{Banfi:2023mhz}.
The same extensions have also enabled the method to be applied in the case of neutrino-induced collisions~\cite{FerrarioRavasio:2024kem}, which now allow for the fully exclusive simulation of (ultra) high-energy neutrino-nucleon collisions.
These new (publicly available) tools improve the precision of fully exclusive simulations of lepton-hadron collisions. This has direct implications for past experiments such as HERA, ongoing measurements at forward detectors at the LHC (e.g. Faser$\nu$, FPF) and large volume neutrino experiments (IceCube, KM3NeT, Baikal), as well as forthcoming colliders such as the EIC and proposed next-generation neutrino detectors.
%
\subsubsection{Strong-coupling constant determination}
\begin{Namen}
P. Nason, G. Zanderighi
\end{Namen}
%
\subsubsection{Time-like matching conditions at the threshold}
\begin{Namen}
C. Biello
\end{Namen}
Threshold conditions are matching ingredients in the DGLAP evolution within the Variable Flavour Number Scheme, governing the effects of crossing a heavy-quark threshold. They represent the final missing process-independent component needed for the NNLO extraction of Fragmentation Functions (FFs), which are time-like non-perturbative objects enabling predictions for identified hadrons. In~\citere{Biello:2024zti}, we have extended the formalism along the lines of~\citere{Cacciari:2005ry} for deriving the matching conditions at NNLO QCD, employing matrix elements from electron-positron annihilation. At this perturbative order, light-quark to hadron FFs require a threshold correction, which we have analytically computed. Future studies will focus on completing the set by deriving threshold conditions for the fragmentation of a gluon or a heavy quark. These investigations are useful not only for achieving formal accuracy in FF fits but also for obtaining deeper insights into the fundamental structure of hadrons. In this context, the known NNLO space-like threshold conditions have been used in providing evidence for intrinsic charm in the proton. These studies showed a non-vanishing charm-flavour wave-function component that is not perturbatively generated by DGLAP evolution to LHC energies, given the threshold conditions at the heavy-quark mass~\cite{Ball:2022qks}.
%
\subsubsection{Mass power corrections for fragmentation functions}
\begin{Namen}
F. Ahmadova, R. Gauld
\end{Namen}
The theoretical description for scattering processes in which a heavy-flavour hadron is identified are typically based on either a massive or a massless description of the quark fragmentation process. These approaches are applicable at either low- or high-energy scales, and for very simple cases a combination of these methods are possible.
On-going work in the group also aims to establish a fully differential description of identified hadron production in a general mass variable flavour number scheme. These developments allow for unified description of identified hadron production in complicated scattering processes with jets and identified hadrons.
These advancements will have important implications for LHC measurements involving heavy-flavour jets, where the standard experimental approach of flavour tagging use the kinematics of identified heavy-flavour hadrons to establish the flavour quantum numbers of jets.
%
\subsubsection{Tetraquarks}
\begin{Namen}
C. Wang
\end{Namen}
%
\subsubsection{Neutrino content of the muon}
\begin{Namen}
F. Garosi
\end{Namen}
\printbibliography[heading=subbibliography]
\end{refsection}

%%%%%%%%%%%%%%%%%%%%%%%%%%%%%%%%%%%%%%%%%%%%%%%%%%%%%%%%%%%%%%%%%%%%%%
\subsection{Beyond Standard Model seaches}
%%%%%%%%%%%%%%%%%%%%%%%%%%%%%%%%%%%%%%%%%%%%%%%%%%%%%%%%%%%%%%%%%%%%%%
\begin{refsection}
Searches for physics beyond the Standard Model (BSM) physics remain a major goal of collider physics experiments.
The sensitivity of many of the conducted searches rely on the availability of precise simulations of signal and background processes, in particular those aiming to observe indirect signals of BSM in terms of (small) deviations of known SM processes or signals of missing energy. 
The development of such tools are essential for the interpretation of collider data, but can also play a crucial role in the design of future experiments~\cite{Bechtle:2024atq}.
%
\subsubsection{Polarised NLO+PS predictions in SMEFT}
\begin{Namen}
J. Linder, G. Pelliccioli, G. Zanderighi
\end{Namen}
%
\subsubsection{MEM method in Powheg and \minnlo{}}
\begin{Namen}
U. Haisch, J. Linder, M. Wiesemann, G. Zanderighi
\end{Namen}
%
\subsubsection{NNLO+PS VH}
\begin{Namen}
R. Gauld, L. Schnell, U. Haisch
\end{Namen}
%
\subsubsection{Z+jet SMEFT}
\begin{Namen}
R. Gauld, U. Haisch, J. Weiss
\end{Namen}
%
\subsubsection{New collider proposal for dark matter studies}
\begin{Namen}
R. Gauld
\end{Namen}
The nature of Dark Matter remains an outstanding question in elementary particle physics, with searches only producing negative (or irreproducible) results.
This suggests to conduct experiments which can probe currently unexplored regions of the parameter space of Dark Matter models.
The Lohengrin experiment is an example of such a proposal~\cite{Bechtle:2024atq} which aims to perform the fixed-target missing momentum based technique to search for dark-sector particles (the ~GeV energy electron beam is provided by the ELSA Accelerator in Bonn).
A simulation tool (lohengrin++) has been developed jointly at the MPP (with Uni. Bonn) for the scattering of electrons and nuclear targets for both signals of (light) Dark Matter as well as SM background processes. This tool has already played an important role in the detector design (feasibility study) and will continue to be developed as planning for the experiment continues. An example of a (differential) Signal-to-Background study for a 10~MeV Dark Photon produced in the collision of a 3.2~GeV electron beam incident on Tungsten is shown in Fig.~{X}.
%
\subsubsection{$b\rightarrow s \gamma$ corrections for the physical value of the charm mass}
\begin{Namen}
M. Niggetiedt
\end{Namen}
%
\printbibliography[heading=subbibliography]
\end{refsection}


% ----------------------------------------------------------------------
\clearpage
\onecolumn
% ----------------------------------------------------------------------
\end{document}
% ----------------------------------------------------------------------
