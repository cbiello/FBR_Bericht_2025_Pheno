% -*- coding:utf-8 -*-
% vi:encoding=utf-8:
% !TEX encoding = UTF-8 Unicode
%
% This file should be utf8 encoded so that these characters render as
% umlauts: ÄÖÜäöüß
%
%
% FBR_Bericht_2025.tex
%
% Fachbeiratsbericht 2025
% 
% PHENO SECTION

\documentclass{FBR_Bericht_2025}

\setcounter{secnumdepth}{5}
\setcounter{tocdepth}{3}

%\renewcommand{\thechapter}{\Roman{chapter}}
%\renewcommand{\thesection}{\arabic{section}}

\usepackage{physics}

\usepackage[backend=biber,sorting=none]{biblatex}

\addbibresource{Pheno.bib}
\input{inc/definitions.tex}

\begin{document}

\onecolumn
\pdfbookmark[1]{Table of Contents}{table_of_contents}
\tableofcontents
\cleardoublepage

\twocolumn

% ----------------------------------------------------------------------
\chapter{Research Activities -- Theory}
% ----------------------------------------------------------------------
\section[Phenomenology]{Novel computational techniques in particle physics and phenomenological applications}
% ----------------------------------------------------------------------
\begin{Namen}
Director: Prof. Dr. G. Zanderighi
\end{Namen}
Introduction of the group.

%%%%%%%%%%%%%%%%%%%%%%%%%%%%%%%%%%%%%%%%%%%%%%%%%%%%%%%%%%%%%%%%%%%%%%
\subsection{Novel event simulations for the hadron--hadron collisions}
%%%%%%%%%%%%%%%%%%%%%%%%%%%%%%%%%%%%%%%%%%%%%%%%%%%%%%%%%%%%%%%%%%%%%%
\begin{refsection}
Full fledged simulations of hadron-level events build the theoretical 
core of any experimental analyses performed at colliders.
By closing the gap between the measured data in the experimental
detectors and the theoretical predictions based on quantum-field theory
(QFT) computations, event generators are one of central tools developed
in particle theory. In addition, their predictions provide a realistic description
of any infrared-safe observable in comparison to data.

To keep up with the steadily decreasing experimental uncertainties, it is 
mandatory to develop event simulations at the highest possible accuracy.
This can be achieved by improving the perturbative all-order description 
of parton showers, on the one hand, and by including higher-order corrections
in the event simulations. This will allows us ultimately to observe even smallest
deviations from SM predictions and paves the way towards new-physics phenomena.

Event generators with next-to-leading order (NLO) accuracy are the standard since 
several years. However, there has been a substantial progress in achieving next-to-NLO
(NNLO) event simulations in the past years, which have been driven by our research
group. The \minnlo{} method to match NNLO predictions with parton showers (NNLO+PS)
was introduced by us in 2019 for $2\to 1$ colour-singlet 
processes \cite{Monni2019:whf,MonniXXX}, generalized $2\to 2$ reactions 
(and beyond) \cite{Lombardi:} and even extended to top-quark pair production \cite{Mazzitelli, Mazzitelli}. With the latter work \minnlo{} became the first 
(and still only) NNLO+PS method to deal with colour charges in initial and final state.
%
\subsubsection{NNLO+PS predictions for B-hadron and b-jet production}
B-hadron and b-jet at NNLO+PS and b-jet algorithms
\begin{Namen}
R. Gauld, J. Mazzitelli, A. Ratti, M. Wiesemann, G. Zanderighi
\end{Namen}
%
\subsubsection{D-meson processes at NNLO+PS \mbox{and prompt neutrino background}}
\begin{Namen}
R. Gauld, T. Giani, A. Mahr, A. Ratti, M. Wiesemann, G. Zanderighi
\end{Namen}
%
\subsubsection[\minnlo{} method for heavy-quark plus colour-singlet production and application to $b\bar{b}Z$ process]{\minnlo{} method for heavy-quark plus colour-singlet production and the \boldmath{$b\bar{b}Z$} process}
\begin{Namen}
M. Wiesemann
\end{Namen}
%
\subsubsection{Associated Higgs production with bottom quarks: a flavour-scheme study at NNLO+PS}
\begin{Namen}
C. Biello, R. Gauld, A. Sankar, M. Wiesemann, G. Zanderighi
\end{Namen}
Example of citation~\cite{Biello:2024pgo} or~\citere{Biello:2024pgo}.
Example of a figure in~\fig{phenofig:bbH}.
\begin{figure}[phtb]
\begin{center}
\includegraphics[width=0.95\linewidth]{plots/bbH__ptHspectrum.pdf}
\caption{Example plot.}
\label{phenofig:bbH}
\end{center}
\end{figure}
%
\subsubsection{Off-shell top-quark pair production at NNLO+PS}
\begin{Namen}
C. Biello, C. Signorile-Signorile, M. Wiesemann, G. Zanderighi
\end{Namen}
%
\subsubsection[\minnlo{} method for jet processes using $N$-jettiness]{\minnlo{} method for jet processes using \boldmath{$N$}-jettiness}
\begin{Namen}
M. Ebert, M. Wiesemann, G. Zanderighi, S. Zanoli
\end{Namen}
%
\subsubsection{Inclusion of electroweak corrections in \minnlo{}}
\begin{Namen}
G. Pelliccioli, M. Wiesemann, G. Zanderighi, S. Zanoli
\end{Namen}
%
\subsubsection{Polarized NLO+PS predictions and quantum info}
\begin{Namen}
G. Pelliccioli, G. Zanderighi
\end{Namen}
%
\printbibliography[heading=subbibliography]
\end{refsection}

\subsection{Fixed-order predictions for the hadron--hadron collisions}
%%%%%%%%%%%%%%%%%%%%%%%%%%%%%%%%%%%%%%%%%%%%%%%%%%%%%%%%%%%%%%%%%%%%%%
\begin{refsection}
Section for predictions at hadron-hadron colliders at fixed order.
%
\subsubsection{Top-mass renormalization in ttbar @ NNLO}
\begin{Namen}
J. Mazzitelli
\end{Namen}
%
\subsubsection{Wbb @ NNLO}
\begin{Namen}
J. Mazzitelli
\end{Namen}
%
\subsubsection{Z+c-jet NNLO QCD}
\begin{Namen}
R. Gauld
\end{Namen}
%
\subsubsection{Off-shell NLO predictions in tt(+X) processes}
\begin{Namen}
G. Pelliccioli
\end{Namen}
%
\printbibliography[heading=subbibliography]
\end{refsection}


%%%%%%%%%%%%%%%%%%%%%%%%%%%%%%%%%%%%%%%%%%%%%%%%%%%%%%%%%%%%%%%%%%%%%%
\subsection[Pushing the precision in Higgs studies]{Pushing the precision in Higgs studies}
%%%%%%%%%%%%%%%%%%%%%%%%%%%%%%%%%%%%%%%%%%%%%%%%%%%%%%%%%%%%%%%%%%%%%%
\begin{refsection}
Space for a nice introduction.
%
\subsubsection{VBF $H\rightarrow b\bar{b}$ production}
\begin{Namen}
A. Behring, G. Zanderighi
\end{Namen}
%
\subsubsection{Two-loop amplitudes for Higgs plus jet}
\begin{Namen}
U. Haisch, M. Niggetiedt
\end{Namen}
%
\subsubsection{Exact top-quark mass dependence in Higgs production}
\begin{Namen}
M. Niggetiedt
\end{Namen}
%
\subsubsection{Higgs predictions with bottom-quark mass effects}
\begin{Namen}
M. Niggetiedt
\end{Namen}
%
\subsubsection{Next-to-soft threshold in \bbH{}}
\begin{Namen}
A. Sankar
\end{Namen}
%
\subsubsection{Rapidity distribution of pseudoscalar Higgs}
\begin{Namen}
A. Sankar
\end{Namen}
%
\subsubsection{Di-Higgs production at NNLO+PS}
\begin{Namen}
F. Garosi, S. Kumar, M. Wiesemann, G. Zanderighi
\end{Namen}
%
\printbibliography[heading=subbibliography]
\end{refsection}

%%%%%%%%%%%%%%%%%%%%%%%%%%%%%%%%%%%%%%%%%%%%%%%%%%%%%%%%%%%%%%%%%%%%%%
\subsection{Tools and methods for higher-order predictions}
%%%%%%%%%%%%%%%%%%%%%%%%%%%%%%%%%%%%%%%%%%%%%%%%%%%%%%%%%%%%%%%%%%%%%%
\begin{refsection}
Space for a nice introduction.
%
\subsubsection{LASS: a subtraction scheme method at NNLO}
\begin{Namen}
G. Pelliccioli, A. Ratti, C. Signorile-Signorile
\end{Namen}
Include formulation and Strongly-ordered infrared counterterms from factorisation.
%
\subsubsection{New formulation of Nested Soft-Collinear Subtraction Scheme}
\begin{Namen}
C. Signorile-Signorile
\end{Namen}
%
\subsubsection{Four-loop renormalisation of pseudoscalar operators}
\begin{Namen}
M. Niggetiedt
\end{Namen}
%
\subsubsection{Soft function at N3LO}
\begin{Namen}
M. Delto, C. Wang
\end{Namen}
%
\subsubsection{Reclassifying Feynman Integrals as Special Functions}
\begin{Namen}
C. Wang
\end{Namen}
%
\printbibliography[heading=subbibliography]
\end{refsection}

%%%%%%%%%%%%%%%%%%%%%%%%%%%%%%%%%%%%%%%%%%%%%%%%%%%%%%%%%%%%%%%%%%%%%%
\subsection{Not only proton-proton collisions}
%%%%%%%%%%%%%%%%%%%%%%%%%%%%%%%%%%%%%%%%%%%%%%%%%%%%%%%%%%%%%%%%%%%%%%
\begin{refsection}
Space for a nice introduction and give me a better title.
%
\subsubsection{NNLO+PS prediction for di-jet production at lepton colliders}
\begin{Namen}
F. Koenig, R. Schorer, M. Wiesemann, G. Zanderighi
\end{Namen}
%
\subsubsection{NLO+PS predictions for charged-lepton and neutrino induced DIS}
\begin{Namen}
R. Gauld, G. Zanderighi
\end{Namen}
%
\subsubsection{Strong-coupling constant determination}
\begin{Namen}
P. Nason, G. Zanderighi
\end{Namen}
%
\subsubsection{Time-like matching conditions at the threshold}
\begin{Namen}
C. Biello
\end{Namen}
Threshold conditions are matching ingredients in the DGLAP evolution within the Variable Flavour Number Scheme, governing the effects of crossing a heavy-quark threshold. They represent the final missing process-independent component needed for the NNLO extraction of Fragmentation Functions (FFs), which are time-like non-perturbative objects enabling predictions for identified hadrons. In~\citere{Biello:2024zti}, we have extended the formalism along the lines of~\citere{Cacciari:2005ry} for deriving the matching conditions at NNLO QCD, employing matrix elements from electron-positron annihilation. At this perturbative order, light-quark to hadron FFs require a threshold correction, which we have analytically computed. Future studies will focus on completing the set by deriving threshold conditions for the fragmentation of a gluon or a heavy quark. These investigations are useful not only for achieving formal accuracy in FF fits but also for obtaining deeper insights into the fundamental structure of hadrons. In this context, the known NNLO space-like threshold conditions have been used in providing evidence for intrinsic charm in the proton. These studies showed a non-vanishing charm-flavour wave-function component that is not perturbatively generated by DGLAP evolution to LHC energies, given the threshold conditions at the heavy-quark mass~\cite{Ball:2022qks}.
%
\subsubsection{Mass power corrections for fragmentation functions}
\begin{Namen}
F. Ahmadova, R. Gauld
\end{Namen}
%
\subsubsection{Tetraquarks}
\begin{Namen}
C. Wang
\end{Namen}
%
\subsubsection{Neutrino content of the muon}
\begin{Namen}
F. Garosi
\end{Namen}
\printbibliography[heading=subbibliography]
\end{refsection}

%%%%%%%%%%%%%%%%%%%%%%%%%%%%%%%%%%%%%%%%%%%%%%%%%%%%%%%%%%%%%%%%%%%%%%
\subsection{Beyond Standard Model seaches}
%%%%%%%%%%%%%%%%%%%%%%%%%%%%%%%%%%%%%%%%%%%%%%%%%%%%%%%%%%%%%%%%%%%%%%
\begin{refsection}
Space for a nice introduction.
%
\subsubsection{Polarised NLO+PS predictions in SMEFT}
\begin{Namen}
J. Linder, G. Pelliccioli, G. Zanderighi
\end{Namen}
%
\subsubsection{MEM method in Powheg and \minnlo{}}
\begin{Namen}
U. Haisch, J. Linder, M. Wiesemann, G. Zanderighi
\end{Namen}
%
\subsubsection{NNLO+PS VH}
\begin{Namen}
R. Gauld, L. Schnell, U. Haisch
\end{Namen}
%
\subsubsection{Z+jet SMEFT}
\begin{Namen}
R. Gauld, U. Haisch, J. Weiss
\end{Namen}
%
\subsubsection{New collider proposal for dark matter studies}
\begin{Namen}
R. Gauld
\end{Namen}
%
\subsubsection{$b\rightarrow s \gamma$ corrections for the physical value of the charm mass}
\begin{Namen}
M. Niggetiedt
\end{Namen}
%
\printbibliography[heading=subbibliography]
\end{refsection}


% ----------------------------------------------------------------------
\clearpage
\onecolumn
% ----------------------------------------------------------------------
\end{document}
% ----------------------------------------------------------------------
