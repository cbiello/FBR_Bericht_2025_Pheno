% -*- coding:utf-8 -*-
% vi:encoding=utf-8:
% !TEX encoding = UTF-8 Unicode
%
% This file should be utf8 encoded so that these characters render as
% umlauts: ÄÖÜäöüß
%
%
% FBR_Bericht_2025.tex
%
% Fachbeiratsbericht 2025
% 
% PHENO SECTION

\documentclass{FBR_Bericht_2025}

\setcounter{secnumdepth}{5}
\setcounter{tocdepth}{3}

%\renewcommand{\thechapter}{\Roman{chapter}}
%\renewcommand{\thesection}{\arabic{section}}

\usepackage{physics}

\usepackage[backend=biber,sorting=none]{biblatex}

\addbibresource{Pheno.bib}
\input{inc/definitions.tex}

\begin{document}

\onecolumn
\pdfbookmark[1]{Table of Contents}{table_of_contents}
\tableofcontents
\cleardoublepage

\twocolumn

% ----------------------------------------------------------------------
\chapter{Research Activities -- Theory}
% ----------------------------------------------------------------------
\section[Phenomenology]{Novel computational techniques in particle physics and phenomenological applications}
% ----------------------------------------------------------------------
\begin{Namen}
Director: Prof. Dr. G. Zanderighi
\end{Namen}
Introduction of the group.

%%%%%%%%%%%%%%%%%%%%%%%%%%%%%%%%%%%%%%%%%%%%%%%%%%%%%%%%%%%%%%%%%%%%%%
\subsection[Event generators at the LHC]{Event generators at the Large Hadron Collider}
%%%%%%%%%%%%%%%%%%%%%%%%%%%%%%%%%%%%%%%%%%%%%%%%%%%%%%%%%%%%%%%%%%%%%%
\begin{refsection}
Section for predictions matched with parton shower in hadron-hadron colliders.
%
\subsubsection{B-hadron and jet algorithms}
\begin{Namen}
R. Gauld, A. Ratti, M. Wiesemann, G. Zanderighi
\end{Namen}
%
\subsubsection{Charm-quark pair production and netrino physics}
\begin{Namen}
R. Gauld, T. Giani, A. Mahr, A. Ratti, M. Wiesemann, G. Zanderighi
\end{Namen}
%
\subsubsection{Exploring a new class of NNLO+PS predictions with bbZ}
\begin{Namen}
M. Wiesemann
\end{Namen}
%
\subsubsection{Higgs production in association with a bottom-quark pair: a flavour-scheme study}
\begin{Namen}
C. Biello, R. Gauld, A. Sankar, M. Wiesemann, G. Zanderighi
\end{Namen}
Example of citation~\cite{Biello:2024pgo} or~\citere{Biello:2024pgo}.
Example of a figure in~\fig{phenofig:bbH}.
\begin{figure}[phtb]
\begin{center}
\includegraphics[width=0.95\linewidth]{plots/bbH__ptHspectrum.pdf}
\caption{Example plot.}
\label{phenofig:bbH}
\end{center}
\end{figure}
%
\subsubsection{Off-shell effects in top-quark pair production}
\begin{Namen}
C. Biello, C. Signorile-Signorile, M. Wiesemann, G. Zanderighi
\end{Namen}
%
\subsubsection{Off-shell studies in tt(+X) processes}
\begin{Namen}
G. Pelliccioli
\end{Namen}
%
\subsubsection{N-jettines formulation of \minnlo{}}
\begin{Namen}
M. Ebert, M. Wiesemann, G. Zanderighi, S. Zanoli
\end{Namen}
%
\subsubsection{EW NLO+PS}
\begin{Namen}
G. Pelliccioli, M. Wiesemann, G. Zanderighi, S. Zanoli
\end{Namen}
%
\subsubsection{Polirised NLO+PS predictions and quantum info}
\begin{Namen}
G. Pelliccioli, G. Zanderighi
\end{Namen}
%
\subsubsection{Di-Higgs production at NNLO+PS}
\begin{Namen}
F. Garosi, M. Wiesemann, G. Zanderighi
\end{Namen}
%
\printbibliography[heading=subbibliography]
\end{refsection}

%%%%%%%%%%%%%%%%%%%%%%%%%%%%%%%%%%%%%%%%%%%%%%%%%%%%%%%%%%%%%%%%%%%%%%
\subsection[Pushing the precision in Higgs studies]{Pushing the precision in Higgs studies}
%%%%%%%%%%%%%%%%%%%%%%%%%%%%%%%%%%%%%%%%%%%%%%%%%%%%%%%%%%%%%%%%%%%%%%
\begin{refsection}
Space for a nice introduction.
%
\subsubsection{VBF $H\rightarrow b\bar{b}$ production}
\begin{Namen}
A. Behring, G. Zanderighi
\end{Namen}
%
\subsubsection{Two-loop amplitudes for Higgs plus jet}
\begin{Namen}
U. Haisch, M. Niggetiedt
\end{Namen}
%
\subsubsection{Exact top-quark mass dependence in Higgs production}
\begin{Namen}
M. Niggetiedt
\end{Namen}
%
\subsubsection{Higgs predictions with bottom-quark mass effects}
\begin{Namen}
M. Niggetiedt
\end{Namen}
%
\subsubsection{Next-to-soft threshold in \bbH{}}
\begin{Namen}
A. Sankar
\end{Namen}
%
\subsubsection{Rapidity distribution of pseudoscalar Higgs}
\begin{Namen}
A. Sankar
\end{Namen}
%
\printbibliography[heading=subbibliography]
\end{refsection}

%%%%%%%%%%%%%%%%%%%%%%%%%%%%%%%%%%%%%%%%%%%%%%%%%%%%%%%%%%%%%%%%%%%%%%
\subsection{Tools and methods for higher-order predictions}
%%%%%%%%%%%%%%%%%%%%%%%%%%%%%%%%%%%%%%%%%%%%%%%%%%%%%%%%%%%%%%%%%%%%%%
\begin{refsection}
Space for a nice introduction.
%
\subsubsection{LASS: a subtraction scheme method at NNLO}
\begin{Namen}
G. Pelliccioli, A. Ratti, C. Signorile-Signorile
\end{Namen}
Include formulation and Strongly-ordered infrared counterterms from factorisation.
%
\subsubsection{New formulation of Nested Soft-Collinear Subtraction Scheme}
\begin{Namen}
C. Signorile-Signorile
\end{Namen}
%
\subsubsection{Four-loop renormalisation of pseudoscalar operators}
\begin{Namen}
M. Niggetiedt
\end{Namen}
%
\subsubsection{Soft function at N3LO}
\begin{Namen}
M. Delto, C. Wang
\end{Namen}
%
\subsubsection{Reclassifying Feynman Integrals as Special Functions}
\begin{Namen}
C. Wang
\end{Namen}
%
\printbibliography[heading=subbibliography]
\end{refsection}

%%%%%%%%%%%%%%%%%%%%%%%%%%%%%%%%%%%%%%%%%%%%%%%%%%%%%%%%%%%%%%%%%%%%%%
\subsection{Not only proton-proton collisions}
%%%%%%%%%%%%%%%%%%%%%%%%%%%%%%%%%%%%%%%%%%%%%%%%%%%%%%%%%%%%%%%%%%%%%%
\begin{refsection}
The application and development of new computational methods and simulation tools has also been pursued for different collider environments, such as those from lepton-hadron or lepton-lepton collisions.
%
Scattering processes with an initial-state lepton or leptons introduce different theoretical challenges as compared to that of hadron-hadron collisions, but also offer exciting opportunities to study the properties of the strong force in a relatively clean environment.
%
Such developments allow to re-visit the available data from colliders such as HERA and LEP with more precise theoretical simulations and inputs, increase the precision of simulation tools available for active experiments (e.g. IceCube and KM3NeT), and also provide a new level of theoretical precision for future experiments such as the EIC or a potential high-energy $e^+e^-$ collider. 
%
\subsubsection{NNLO+PS prediction for di-jet production at lepton colliders}
\begin{Namen}
F. Koenig, R. Schorer, M. Wiesemann, G. Zanderighi
\end{Namen}
%
\subsubsection{NLO+PS predictions for charged-lepton and neutrino induced DIS}
\begin{Namen}
R. Gauld, G. Zanderighi
\end{Namen}
The POWHEG method of matching fixed-order predictions with parton showers at NLO in QCD has been successfully applied to hadron-hadron collisions for a vast range of processes. As a consequence of the genuinely different kinematics encountered in lepton-hadron collisions, substantial extensions to the POWHEX BOX framework had to be undertaken to enable this method to be applied in this new collision environment~\cite{Banfi:2023mhz}.
The same extensions have also enabled the method to be applied in the case of neutrino-induced collisions~\cite{FerrarioRavasio:2024kem}, which now allow for the fully exclusive simulation of (ultra) high-energy neutrino-nucleon collisions.
These new (publicly available) tools improve the precision of fully exclusive simulations of lepton-hadron collisions. This has direct implications for past experiments such as HERA, ongoing measurements at forward detectors at the LHC (e.g. Faser$\nu$, FPF) and large volume neutrino experiments (IceCube, KM3NeT, Baikal), as well as forthcoming colliders such as the EIC and proposed next-generation neutrino detectors.
%
\subsubsection{Strong-coupling constant determination}
\begin{Namen}
P. Nason, G. Zanderighi
\end{Namen}
%
\subsubsection{Time-like matching conditions at the threshold}
\begin{Namen}
C. Biello
\end{Namen}
Threshold conditions are matching ingredients in the DGLAP evolution within the Variable Flavour Number Scheme, governing the effects of crossing a heavy-quark threshold. They represent the final missing process-independent component needed for the NNLO extraction of Fragmentation Functions (FFs), which are time-like non-perturbative objects enabling predictions for identified hadrons. In~\citere{Biello:2024zti}, we have extended the formalism along the lines of~\citere{Cacciari:2005ry} for deriving the matching conditions at NNLO QCD, employing matrix elements from electron-positron annihilation. At this perturbative order, light-quark to hadron FFs require a threshold correction, which we have analytically computed. Future studies will focus on completing the set by deriving threshold conditions for the fragmentation of a gluon or a heavy quark. These investigations are useful not only for achieving formal accuracy in FF fits but also for obtaining deeper insights into the fundamental structure of hadrons. In this context, the known NNLO space-like threshold conditions have been used in providing evidence for intrinsic charm in the proton. These studies showed a non-vanishing charm-flavour wave-function component that is not perturbatively generated by DGLAP evolution to LHC energies, given the threshold conditions at the heavy-quark mass~\cite{Ball:2022qks}.
%
\subsubsection{Mass power corrections for fragmentation functions}
\begin{Namen}
F. Ahmadova, R. Gauld
\end{Namen}
The theoretical description for scattering processes in which a heavy-flavour hadron is identified are typically based on either a massive or a massless description of the quark fragmentation process. These approaches are applicable at either low- or high-energy scales, and for very simple cases a combination of these methods are possible.
On-going work in the group also aims to establish a fully differential description of identified hadron production in a general mass variable flavour number scheme. These developments allow for unified description of identified hadron production in complicated scattering processes with jets and identified hadrons.
These advancements will have important implications for LHC measurements involving heavy-flavour jets, where the standard experimental approach of flavour tagging use the kinematics of identified heavy-flavour hadrons to establish the flavour quantum numbers of jets.
%
\subsubsection{Tetraquarks}
\begin{Namen}
C. Wang
\end{Namen}
%
\subsubsection{Neutrino content of the muon}
\begin{Namen}
F. Garosi
\end{Namen}
\printbibliography[heading=subbibliography]
\end{refsection}

%%%%%%%%%%%%%%%%%%%%%%%%%%%%%%%%%%%%%%%%%%%%%%%%%%%%%%%%%%%%%%%%%%%%%%
\subsection{Beyond Standard Model seaches}
%%%%%%%%%%%%%%%%%%%%%%%%%%%%%%%%%%%%%%%%%%%%%%%%%%%%%%%%%%%%%%%%%%%%%%
\begin{refsection}
Space for a nice introduction.
%
\subsubsection{Polarised NLO+PS predictions in SMEFT}
\begin{Namen}
J. Linder, G. Pelliccioli, M. Wiesemann, G. Zanderighi
\end{Namen}
%
\subsubsection{NNLO+PS VH}
\begin{Namen}
R. Gauld, L. Schnell, U. Haisch
\end{Namen}
%
\subsubsection{Z+jet SMEFT}
\begin{Namen}
R. Gauld, U. Haisch, J. Weiss
\end{Namen}
%
\subsubsection{New collider proposal for dark matter studies}
\begin{Namen}
R. Gauld
\end{Namen}
%
\subsubsection{$b\rightarrow s \gamma$ corrections for the physical value of the charm mass}
\begin{Namen}
M. Niggetiedt
\end{Namen}
%
\printbibliography[heading=subbibliography]
\end{refsection}


% ----------------------------------------------------------------------
\clearpage
\onecolumn
% ----------------------------------------------------------------------
\end{document}
% ----------------------------------------------------------------------
